%% Except where otherwise noted, content in this documentation is Copyright (c)
%% 2022, RTE (http://www.rte-france.com) and licensed under a
%% CC-BY-4.0 (https://creativecommons.org/licenses/by/4.0/)
%% license. All rights reserved.

\documentclass[a4paper, 12pt]{report}

% Latex setup
\input{../latex_setup.tex}

\begin{document}

\title{\Dynawo-algorithms Introduction Documentation}
\date\today

\maketitle
\tableofcontents

\chapter{Introduction}

\section{What is \Dynawo-algorithms?}

\textbf{\Dynawo-algorithms is a wrapper around  \href{https://dynawo.github.io/}{\underline{\Dynawo}} 
that provides utility algorithms to calculate complex key values of a power system.} 

It can be used with any tool of the \Dynawo toolsuite (DynaFlow, DynaWaltz, DynaSwing, DySym, DynaWave).\\

It provides the following possibilities:
\begin{itemize}
  \item \textbf{Unitary simulations}: simulations of one or several power systems described in a \textbf{.jobs} file 
  (see \href{https://github.com/dynawo/dynawo-algorithms/releases/download/v1.5.0/DynawoAlgorithmsDocumentation.pdf}{\underline{\Dynawo documentation}});
  \item \textbf{Systematic analysis}: simulations of a same base power system subject to different events to assess the global stability;
  \item \textbf{Margin calculation}: simulations of a same base power system with a load variation and subject to different events to compute 
  the maximum load increase in a specific region before the voltage collapses;
  \item \textbf{Load increase}: simulation of a single load variation.
\end{itemize}

\Dynawo-algorithms is licensed under the terms of the \href{http://mozilla.org/MPL/2.0}{\underline{Mozilla Public License, v2.0}}.
The source code is hosted into a \href{https://github.com/dynawo/dynawo-algorithms} {\underline{GitHub repository}}. \\

\section{Changes from previous versions}

\subsection{Changes from v1.5.0}
\underline{General:}

\begin{itemize}
\item Dynawo-algorithms is now available on windows
\end{itemize}

\underline{Systematic Analysis and Margin calculation:}

\begin{itemize}
\item a simulation which diverges and has some criteria not met will be seen with the status DIVERGENCE
\end{itemize}

\subsection{Changes from v1.4.1}
None

\subsection{Changes from v1.4.0}

\underline{New features:}
\begin{itemize}
\item Margin calculation: to avoid running all scenarios when the 100\% load increase fail we now launch the maximum load increase passing before continuing the dichotomy
\end{itemize}

\underline{Bug fix:}
\begin{itemize}
\item Margin calculation with variation: properly recompute the end time of the simulation
\end{itemize}

\subsection{Changes from v1.3.1}
Minor fixes.

\subsection{Changes from v1.3.0}
First release.

\end{document}
