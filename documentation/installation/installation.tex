%% Except where otherwise noted, content in this documentation is Copyright (c)
%% 2022, RTE (http://www.rte-france.com) and licensed under a
%% CC-BY-4.0 (https://creativecommons.org/licenses/by/4.0/)
%% license. All rights reserved.

\documentclass[a4paper, 12pt]{report}

% Latex setup
\input{../latex_setup.tex}

\begin{document}

\title{\Dynawo-algorithms Installation Documentation}
\date\today

\maketitle
\tableofcontents

\chapter{Install procedure}

\Dynawo-algorithm is available on \textbf{Linux}. 
You can either build it from sources or use binaries distributed on Github.
This project adds additional features to the \href{https://github.com/dynawo/dynawo}{\Dynawo} project.

\section{\Dynawo-algorithms binaries distribution}

\Dynawo-algorithms release is available on Github : \href{https://github.com/dynawo/dynawo-algorithms/releases/download/v1.4.1/DynawoAlgorithms_Linux_v1.4.1.zip}{Linux release}.

The packages required to use the distribution are the same as \Dynawo.

\subsection{Using a distribution}

You can use the following commands to download and test the latest distribution:
\begin{lstlisting}[language=bash, breaklines=true, breakatwhitespace=false]
$> curl -L $(curl -s -L -X GET https://api.github.com/repos/dynawo/dynawo-algorithms/releases/latest | grep "DynawoAlgorithms_Linux" | grep url | cut -d '"' -f 4) -o DynawoAlgorithms_Linux_latest.zip
$> unzip DynawoAlgorithms_Linux_latest.zip
$> cd dynawo-algorithms
$> ./dynawo-algorithms.sh CS --input Dynawo_Linux_latest/nrt/data/IEEE14/IEEE14_BlackBoxModels/IEEE14.jobs
$> ./dynawo-algorithms.sh help
\end{lstlisting}

\section{Building requirements}

\Dynawo-algorithms is tested on Linux platforms (Centos, Debian and Ubuntu based) and provided that you can install system packages there should be no problem on any other Linux distribution. 

The requirements to build \Dynawo-algorithms are the same as \Dynawo.

\section[Building Dynawo-algorithms]{Building \Dynawo-algorithms}
\label{Dynawo_algorithms_Installation_Documentation_Building_Dynawo_algorithm}
The first step is to build \Dynawo in a separate folder. Please refer to the \Dynawo documentation to do so.
Then, the following command needs to be launched from the \Dynawo folder.

\begin{lstlisting}[language=bash]
$> ./myEnvDynawo.sh deploy
\end{lstlisting}

This command creates a deploy folder. The path to dynawo deploy is then the path to the subdirectory dynawo in the deploy folder. It is similar to:

\begin{lstlisting}[language=bash]
$> PATH_TO_DYNAWO_DEPLOY=<DYNAWO FOLDER>/deploy/<COMPILER><COMPILER VERSION>/shared/dynawo/
\end{lstlisting}

To build \Dynawo-algorithms you need to clone the github repository and launch the following commands in the source code directory:

\begin{lstlisting}[language=bash]
$> git clone https://github.com/dynawo/dynawo-algorithms.git
$> cd dynawo-algorithms
$> echo '#!/bin/bash
export DYNAWO_ALGORITHMS_HOME=$(cd "$(dirname "${BASH_SOURCE[0]}")" && pwd)
export DYNAWO_HOME=PATH_TO_DYNAWO_DEPLOY

export DYNAWO_LOCALE=en_GB
export DYNAWO_RESULTS_SHOW=true
export DYNAWO_BROWSER=firefox

export DYNAWO_NB_PROCESSORS_USED=1

export DYNAWO_BUILD_TYPE=Release

$DYNAWO_ALGORITHMS_HOME/util/envDynawoAlgorithms.sh $@' > myEnvDynawoAlgorithms.sh
$> chmod +x myEnvDynawoAlgorithms.sh
$> ./myEnvDynawoAlgorithms.sh build
\end{lstlisting}

Below is a description of some environment variables that can be modified in the file \textit{myEnvDynawoAlgorithms.sh}:

\begin{center}
\begin{tabular}{|l|l|}
  \hline
   DYNAWO\_BROWSER & Default browser command \\
  \hline
   DYNAWO\_NB\_PROCESSORS\_USED & Maximum number of cores to use \\
  \hline
   DYNAWO\_BUILD\_TYPE & Build type: Release or Debug \\
  \hline
\end{tabular}
\end{center}

\textbf{Warning}: If you're working behind a proxy make sure you have exported the following proxy environment variables:

\begin{lstlisting}[language=bash]
$> export http_proxy=
$> export https_proxy=
$> export no_proxy=localhost,127.0.0.0/8,::1
$> export HTTP_PROXY=$http_proxy;export HTTPS_PROXY=$https_proxy;export NO_PROXY=$no_proxy;
\end{lstlisting}

\section[Launching Dynawo-algorithms]{Launching \Dynawo-algorithms}

Once you have installed and compiled \Dynawo-algorithms as explained in part \ref{Dynawo_algorithms_Installation_Documentation_Building_Dynawo_algorithm}, 
you can launch a simulation by calling one example:

\begin{lstlisting}[language=bash, breaklines=true, breakatwhitespace=false]
$> ./myEnvDynawoAlgorithms.sh CS --input nrt/data/IEEE14/IEEE14_BlackBoxModels/IEEE14.jobs
\end{lstlisting}

This command launches a simple simulation on the IEEE 14-bus network that should succeed if your installation went well and your compilation finished successfully.

\section{Third parties}

To run a simulation on Linux, \Dynawo-algorithms uses several external libraries that are downloaded and compiled during the building process:
\begin{itemize}
\item \href{https://github.com/gperftools/gperftools} {\underline{gperftools}}, a collection of a high-performance multi-threaded
malloc implementations distributed under the BSD license. \Dynawo-algorithms is currently using the version 2.6.1.

\item \href{https://www.mpich.org/}{\underline{MPICH}}, an implementation of the Message Passing Interface (MPI) standard distributed under a BSD-like license. 
\Dynawo-algorithms is currently using the version 3.4.2.
\end{itemize}

To run a simulation on Windows, \Dynawo-algorithms uses an external librarie that has to be installed before the building process:
\begin{itemize}
\item \href{https://learn.microsoft.com/en-us/message-passing-interface/microsoft-mpi?redirectedfrom=MSDN}{\underline{MSMPI}}, a Microsoft implementation of the Message Passing Interface standard distributed under a MIT license.
\Dynawo-algorithms is currently using the version 10.1.2.
\end{itemize}

In addition to these libraries needed for the simulation process, \Dynawo-algorithms downloads the code for one other library:
\begin{itemize}
\item \href{https://github.com/google/styleguide/tree/gh-pages/cpplint}{\underline{cpplint}}, 
a tool used during \Dynawo-algorithms compilation process to ensure that the C++ files follow the Google\textquotesingle s C++ style.
\end{itemize}

Finally, \Dynawo-algorithms also uses others libraries for the unit testing process or to build its source documentation. 

\end{document}
