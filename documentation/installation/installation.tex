%% Except where otherwise noted, content in this documentation is Copyright (c)
%% 2022, RTE (http://www.rte-france.com) and licensed under a
%% CC-BY-4.0 (https://creativecommons.org/licenses/by/4.0/)
%% license. All rights reserved.

\documentclass[a4paper, 12pt]{report}

% Latex setup
%% Except where otherwise noted, content in this documentation is Copyright (c)
%% 2022, RTE (http://www.rte-france.com) and licensed under a
%% CC-BY-4.0 (https://creativecommons.org/licenses/by/4.0/)
%% license. All rights reserved.

% Latin Modern fam­ily of fonts
\usepackage{lmodern}

\usepackage[english]{babel}

% specify encoding
\usepackage[utf8]{inputenc} % input
\usepackage[T1]{fontenc} % output

% Document structure setup
\usepackage{titlesec} % To change chapter format
\setcounter{tocdepth}{3} % Add subsubsection in Content
\setcounter{secnumdepth}{3} % Add numbering for subsubsection
\setlength{\parindent}{0pt} % No paragraph indentation

% Change title format for chapter
\titleformat{\chapter}{\Huge\bf}{\thechapter}{20pt}{\Huge\bf}

% To add links on page number in Content and hide red rectangle on links
\usepackage[hidelinks, linktoc=all]{hyperref}
\usepackage[nottoc]{tocbibind}  % To add biblio in table of content
\usepackage{textcomp} % For single quote
\usepackage{url} % Allow linebreaks in \url command
\usepackage{listings} % To add code samples

% Default listings parameters
\lstset
{
  aboveskip={1\baselineskip}, % a bit of space above
  backgroundcolor=\color{shadecolor}, % choose the background color
  basicstyle={\ttfamily\footnotesize}, % use font and smaller size \small \footnotesize
  breakatwhitespace=true, % sets if automatic breaks should only happen at whitespace
  breaklines=true, % sets automatic line breaking
  columns=fixed, % nice spacing -> fixed / flexible
  mathescape=false, % escape to latex false
  numbers=left, % where to put the line-numbers
  numberstyle=\tiny\color{gray}, % the style that is used for the line-numbers
  showstringspaces=false, % do not emphasize spaces in strings
  tabsize=4, % number of spaces of a TAB
  texcl=false, % activates or deactivates LaTeX comment lines
  upquote=true % upright quotes
}

% Avoid numbering starting at each chapter for figures
\usepackage{chngcntr}
\counterwithout{figure}{chapter}

\usepackage{tikz} % macro pack­age for cre­at­ing graph­ics
\usepackage{pgfplots} % draws func­tion plots (based on pgf/tikz)
\pgfplotsset{enlarge x limits=false, xlabel={\begin{small}$time$ (s)\end{small}}, height=0.6\textwidth, width=1\textwidth,
yticklabel style={text width={width("$-0.6$")},align=right},
/pgf/number format/precision=4}
\pgfplotstableset{col sep=semicolon}

\usepackage{algorithm} % Add algorithms
\usepackage[noend]{algpseudocode} %  all end ... lines are omitted in algos

\usepackage{amsmath} % Add math­e­mat­i­cal fea­tures
\usepackage{schemabloc} % Add block diagram library (french one)

\usepackage{adjustbox} % Add box for flowchart

\usepackage{booktabs} % for toprule and midrule in tables

\usepackage{tabularx}

\usepackage[nolist]{acronym} % don’t write the list of acronyms.
% Acronyms list
\begin{acronym}
\acro{BDF}{Backward Differentiation Formula}
\acro{BE}{Backward Euler}
\acro{DAE}{Differential Algebraic Equations}
\acro{IDA}{Implicit Differential-Algebraic solver}
\acro{LLNL}{Lawrence Livermore National Lab}
\acro{KINSOL}{Krylov Inexact Newton SOLver}
\acro{NR}{Newton-Raphson}
\acro{PLL}{Phase-Locked Loop}
\acro{SVC}{Static Var Compensator}
\acro{SUNDIALS}{SUite of Nonlinear and DIfferential/ALgebraic equation Solvers}
\end{acronym}

% Syntax highlight
%% Except where otherwise noted, content in this documentation is Copyright (c)
%% 2022, RTE (http://www.rte-france.com) and licensed under a
%% CC-BY-4.0 (https://creativecommons.org/licenses/by/4.0/)
%% license. All rights reserved.

\usepackage{color}

\definecolor{blue}{rgb}{0,0,1}
\definecolor{lightblue}{rgb}{.3,.5,1}
\definecolor{darkblue}{rgb}{0,0,.4}
\definecolor{red}{rgb}{1,0,0}
\definecolor{darkred}{rgb}{.56,0,0}
\definecolor{pink}{rgb}{.933,0,.933}
\definecolor{purple}{rgb}{0.58,0,0.82}
\definecolor{green}{rgb}{0.133,0.545,0.133}
\definecolor{darkgreen}{rgb}{0,.4,0}
\definecolor{gray}{rgb}{.3,.3,.3}
\definecolor{darkgray}{rgb}{.2,.2,.2}
\definecolor{shadecolor}{gray}{0.925}

% **********************************************************************************
% Syntax : Bash (bash)
% **********************************************************************************

\lstdefinelanguage{bash}
{
  keywordstyle=\color{blue},
  morekeywords={
    cd,
    export,
    source},
  numbers=none,
  deletekeywords={jobs}
}

% **********************************************************************************
% Syntax : XML
% **********************************************************************************

\lstdefinelanguage{XML}
{
  morestring=[s][\color{purple}]{"}{"},
  morecomment=[s][\color{green}]{<?}{?>},
  morecomment=[s][\color{green}]{<!--}{-->},
  stringstyle=\color{black},
  identifierstyle=\color{blue},
  keywordstyle=\color{red},
  morekeywords={
    xmlns,
    xsi,
    noNamespaceSchemaLocation,
    type,
    source,
    target,
    version,
    tool,
    transRef,
    roleRef,
    objective,
    eventually}
}

% **********************************************************************************
% Syntax : Modelica (modelica)
% **********************************************************************************
\lstdefinelanguage{Modelica}{
  alsoletter={...},
  morekeywords=[1]{ % types
      Boolean,
      Integer,
      Real},
  keywordstyle=[1]\color{red},
  morekeywords=[2]{ % keywords
    algorithm,
    and,
    annotation,
    assert,
    block,
    class,
    connector,
    constant,
    discrete,
    else,
    elseif,
    elsewhen,
    end,
    equation,
    exit,
    extends,
    external,
    false,
    final,
    flow,
    for,
    function,
    if,
    in,
    inner,
    input,
    import,
    loop,
    model,
    nondiscrete,
    not,
    or,
    outer,
    output,
    package,
    parameter,
    public,
    protected,
    record,
    redeclare,
    replaceable,
    return,
    size,
    terminate,
    then,
    true,
    type,
    when,
    while},
  keywordstyle=[2]\color{darkred},
  morekeywords=[3]{ % functions
    abs,
    acos,
    asin,
    atan,
    atan2,
    Complex,
    connect,
    conj,
    cos,
    cosh,
    cross,
    der,
    edge,
    exp,
    fromPolar,
    imag,
    noEvent,
    pre,
    sign,
    sin,
    sinh,
    sqrt,
    tan,
    tanh},
  keywordstyle=[3]\color{blue},
  morecomment=[l][\color{green}]{//}, % comments
  morecomment=[s][\color{green}]{/*}{*/}, % comments
  morestring=[b][\color{pink}]{'}, % strings
  morestring=[b][\color{pink}]{"}, % strings
}

\usepackage{tikz}
\definecolor{blue}{rgb}{.3,.5,1}
\definecolor{red}{rgb}{1,0,0}
\usetikzlibrary{shapes,arrows}
% Define block styles
\tikzstyle{decision} = [diamond, draw, fill=blue!20, 
    text width=4.5em, text badly centered, node distance=3cm, inner sep=0pt]
\tikzstyle{block} = [rectangle, draw, fill=blue!20, 
    text width=5em, text centered, rounded corners, minimum height=4em]
\tikzstyle{line} = [draw, -latex']
\tikzstyle{cloud} = [draw, ellipse,fill=red!20, node distance=3cm,
    minimum height=2em]
    \usetikzlibrary{calc}


\usepackage{xspace} % Define typography
\usepackage{dirtree}
\newcommand{\Dynawo}[0]{Dyna$\omega$o\xspace}


\begin{document}

\title{\Dynawo-algorithms Installation Documentation}
\date\today

\maketitle
\tableofcontents

\chapter{Install procedure}

\Dynawo-algorithm is available on \textbf{Linux}. 
You can either build it from sources or use binaries distributed on Github.
This project adds additional features to the \href{https://github.com/dynawo/dynawo}{\Dynawo} project.

\section{\Dynawo-algorithms binaries distribution}

To start testing \Dynawo-algorithms you can use binary releases on Github:
\begin{itemize}
\item
  \href{https://github.com/dynawo/dynawo-algorithms/releases/download/v1.7.0/DynawoAlgorithms_Linux_v1.7.0.zip}{Linux}
\item
  \href{https://github.com/dynawo/dynawo-algorithms/releases/download/v1.7.0/DynawoAlgorithms_Windows_v1.7.0.zip}{Windows}
\end{itemize}


The packages required to use the distribution are the same as \Dynawo.

\subsection{Using a distribution}

You can use the following commands to download and test the latest distribution:
\begin{lstlisting}[language=bash, breaklines=true, breakatwhitespace=false]
$> curl -L $(curl -s -L -X GET https://api.github.com/repos/dynawo/dynawo-algorithms/releases/latest | grep "DynawoAlgorithms_Linux" | grep url | cut -d '"' -f 4) -o DynawoAlgorithms_Linux_latest.zip
$> unzip DynawoAlgorithms_Linux_latest.zip
$> cd dynawo-algorithms
$> ./dynawo-algorithms.sh CS --input Dynawo_Linux_latest/nrt/data/IEEE14/IEEE14_BlackBoxModels/IEEE14.jobs
$> ./dynawo-algorithms.sh help
\end{lstlisting}

\section{Building requirements}

\Dynawo-algorithms is tested on Linux platforms (Centos, Debian and Ubuntu based) and provided that you can install system packages there should be no problem on any other Linux distribution. 

The requirements to build \Dynawo-algorithms are the same as \Dynawo.

\section[Building Dynawo-algorithms]{Building \Dynawo-algorithms}
\label{Dynawo_algorithms_Installation_Documentation_Building_Dynawo_algorithm}
The first step is to build \Dynawo in a separate folder. Please refer to the \Dynawo documentation to do so.
Then, the following command needs to be launched from the \Dynawo folder.

\begin{lstlisting}[language=bash]
$> ./myEnvDynawo.sh deploy
\end{lstlisting}

This command creates a deploy folder. The path to dynawo deploy is then the path to the subdirectory dynawo in the deploy folder. It is similar to:

\begin{lstlisting}[language=bash]
$> PATH_TO_DYNAWO_DEPLOY=<DYNAWO FOLDER>/deploy/<COMPILER><COMPILER VERSION>/shared/dynawo/
\end{lstlisting}

To build \Dynawo-algorithms you need to clone the github repository and launch the following commands in the source code directory:

\begin{lstlisting}[language=bash]
$> git clone https://github.com/dynawo/dynawo-algorithms.git
$> cd dynawo-algorithms
$> echo '#!/bin/bash
export DYNAWO_ALGORITHMS_HOME=$(cd "$(dirname "${BASH_SOURCE[0]}")" && pwd)
export DYNAWO_HOME=PATH_TO_DYNAWO_DEPLOY

export DYNAWO_LOCALE=en_GB
export DYNAWO_RESULTS_SHOW=true
export DYNAWO_BROWSER=firefox

export DYNAWO_NB_PROCESSORS_USED=1

export DYNAWO_BUILD_TYPE=Release

$DYNAWO_ALGORITHMS_HOME/util/envDynawoAlgorithms.sh $@' > myEnvDynawoAlgorithms.sh
$> chmod +x myEnvDynawoAlgorithms.sh
$> ./myEnvDynawoAlgorithms.sh build
\end{lstlisting}

Below is a description of some environment variables that can be modified in the file \textit{myEnvDynawoAlgorithms.sh}:

\begin{center}
\begin{tabular}{|l|l|}
  \hline
   DYNAWO\_BROWSER & Default browser command \\
  \hline
   DYNAWO\_NB\_PROCESSORS\_USED & Maximum number of cores to use \\
  \hline
   DYNAWO\_BUILD\_TYPE & Build type: Release or Debug \\
  \hline
\end{tabular}
\end{center}

\textbf{Warning}: If you're working behind a proxy make sure you have exported the following proxy environment variables:

\begin{lstlisting}[language=bash]
$> export http_proxy=
$> export https_proxy=
$> export no_proxy=localhost,127.0.0.0/8,::1
$> export HTTP_PROXY=$http_proxy;export HTTPS_PROXY=$https_proxy;export NO_PROXY=$no_proxy;
\end{lstlisting}

\section[Launching Dynawo-algorithms]{Launching \Dynawo-algorithms}

Once you have installed and compiled \Dynawo-algorithms as explained in part \ref{Dynawo_algorithms_Installation_Documentation_Building_Dynawo_algorithm}, 
you can launch a simulation by calling one example:

\begin{lstlisting}[language=bash, breaklines=true, breakatwhitespace=false]
$> ./myEnvDynawoAlgorithms.sh CS --input nrt/data/IEEE14/IEEE14_BlackBoxModels/IEEE14.jobs
\end{lstlisting}

This command launches a simple simulation on the IEEE 14-bus network that should succeed if your installation went well and your compilation finished successfully.

\section{Third parties}

To run a simulation on Linux, \Dynawo-algorithms uses several external libraries that are downloaded and compiled during the building process:
\begin{itemize}
\item \href{https://github.com/gperftools/gperftools} {\underline{gperftools}}, a collection of a high-performance multi-threaded
malloc implementations distributed under the BSD license. \Dynawo-algorithms is currently using the version 2.6.1.

\item \href{https://www.mpich.org/}{\underline{MPICH}}, an implementation of the Message Passing Interface (MPI) standard distributed under a BSD-like license. 
\Dynawo-algorithms is currently using the version 3.4.2.
\end{itemize}

To run a simulation on Windows, \Dynawo-algorithms uses an external librarie that has to be installed before the building process:
\begin{itemize}
\item \href{https://learn.microsoft.com/en-us/message-passing-interface/microsoft-mpi?redirectedfrom=MSDN}{\underline{MSMPI}}, a Microsoft implementation of the Message Passing Interface standard distributed under a MIT license.
\Dynawo-algorithms is currently using the version 10.1.2.
\end{itemize}

In addition to these libraries needed for the simulation process, \Dynawo-algorithms downloads the code for one other library:
\begin{itemize}
\item \href{https://github.com/google/styleguide/tree/gh-pages/cpplint}{\underline{cpplint}}, 
a tool used during \Dynawo-algorithms compilation process to ensure that the C++ files follow the Google\textquotesingle s C++ style.
\end{itemize}

Finally, \Dynawo-algorithms also uses others libraries for the unit testing process or to build its source documentation. 

\end{document}
