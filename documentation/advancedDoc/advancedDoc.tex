%% Except where otherwise noted, content in this documentation is Copyright (c)
%% 2022, RTE (http://www.rte-france.com) and licensed under a
%% CC-BY-4.0 (https://creativecommons.org/licenses/by/4.0/)
%% license. All rights reserved.

\documentclass[a4paper, 12pt]{report}

% Latex setup
\input{../latex_setup.tex}

\begin{document}

\title{\Dynawo-algorithms Advanced Documentation}
\date\today

\maketitle
\tableofcontents

\chapter{Advanced documentation}

This chapter contains documentation on advanced features for readers that would like to have a deep look in \Dynawo-algorithms. It will explain:
\begin{itemize}
\item \Dynawo code organization (\ref{Dynawo_Algorithms_Advanced_Documentation_Code_Organization})
\end{itemize}

\section{Code organization}
\label{Dynawo_Algorithms_Advanced_Documentation_Code_Organization}

The \Dynawo-algorithms source code \href{https://github.com/dynawo/dynawo-algorithms.git}
{\underline{repository}} is organized as follows:
\dirtree{%
.1 cmake.
.1 cpplint.
.1 documentation.
.1 doxygen.
.1 nrt.
.1 sources.
.3 API.
.3 Common.
.3 Launcher.
.1 util.
}
\begin{itemize}
\item the cmake folder contains all the files related to the general
configuration of the compilation process;
\item the cpplint directory contains the Python scripts needed to use cpplint;
\item the documentation directory contains the different latex documents used to
create this documentation;
\item the doxygen folder contains the \Dynawo-algorithms settings to build the source code
documentation;
\item the nrt directory contains the testsuite used to check the correct
behaviour of the tool. They also serve as examples of the 
tool behaviour and main principles;
\item the sources directory contains all the code related to \Dynawo-algorithms itself. This is the most important directory;
\item the util directory contains some utilities code.
\end{itemize}

The source code is divided into different subdirectories corresponding to the different parts of the \Dynawo-algorithms tool. They are:
\begin{itemize}
\item the API directory contains the code related to all the Input/Output
files;
\item the Common directory contains all the code dealing with common feature and methods as well as the dictionaries for logs;
\item the Launcher folder contains the main algorithms;
\end{itemize}

\end{document}
